\section{Perspektywy rozwoju}
W dzisiejszych czasach nie jest możliwe wyobrażenie sobie rolnictwa precyzyjnego bez dokładnie wyznaczanej pozycji przestrzennej maszyn,
zatem rozwój tej nabierającej obecnie coraz większego znaczenia dziedziny rolnictwa upatruje się w rozwoju szeroko pojętej nawigacji.
Wielkie nadzieje wiąże się zatem z ukończeniem budowy i osiągnięciem pełni operacyjności nowych systemów nawigacji satelitarnej - GALILEO oraz modernizacją 
systemów już istniejących - GPS, GLONASS. 
W pełni operacyjny system GALILEO w załorzeniach ma zapewnić minimum sześć satelitów widocznych globalnie powyżej 10\degree nad horyzontem. Co będzie stanowiło 
wzrost w liczbie dostępnych satelitów na poziomie ponad 50\%. Większa liczba widocznych satelitów to więcej obserwacji, z zatem z czysto statystycznego punktu 
widzenia większa dokładność wyznaczanej pozycji zwłaszcza w trudnych warunkach terenowych jak lasy czy zadrzewienia śródpolne. Ponadto satelity systemu GALILEO 
będą emitować sygnały modulowane w trzech pasmach częstotliwości radiowych, co pozwoli za pomocą odpowiednich kombinacji liniowych zarówno w przypadku obserwacji fazowych
jak i kodowych na jeszcze skuteczniejszą eliminację wpływu refrekcji jonosferycznej. Dodatkowy system to także lepsza możliwość wykrywania nieprawidłowości 
w danych GNSS oraz wszelkiego rodzaju awarii przez co zwiekszona będzie niezawodność oraz odporność na zakłócenia rozwiązania nawigacyjnego.
Modernizacja operacyjnych obecnie systemów GPS oraz GLONASS w postaci wyniesienia na orbitę satelitów odpowiednio typów IIF oraz III dla GPS oraz trzeciej generacji 
w przypadku systemu GLONASS, również zwiększy moc oraz jakość ich sygnałów. Ponadto w załorzeniach oba systemy również wprowadzą trzecie pasmo częstotliwości 
odpowiednio L5\footnote{Trzecia częstotliwość modulacji sygnału GPS jest już dostępna eksperymentalnie od kwietnia roku 2014} oraz G3, które podobnie jak w przypadku GALILEO
zwiększy możliwości eliminacji niekorzystnego wpływu refrakcji jonosferycznej z obserwacji. W ogólności wprowadzenie trzeciego pasma częstotliwości umożliwi odpowiednie 
kombinacje liniowe obserwacji charakteryzujące się znacznie dłuższą wypadkową długością fali, co przyczyni się do szybszego\footnote{
zmniejszy się czas (covergence time) potrzebny na uzyskanie stabilnego rowiązania} oraz dokładniejszego wyznaczania nieoznaczoności w precyzyjnych pomiarach absolutnych PPP.
Z punktu widzenia rolnictwa precyzyjnego z powyższych rozważań najistotniejsza wydaje się poprawa dokładności zwłaszcza rozwiązań opartych na obserwacjach kodowych oraz 
umożliwienie szybkiego znajdowania stabilnego roziązania nieoznaczoności w pomiarach absolutnych. Szacuje się, że samo wprowadzenie trzeciej częstotliwości umożliwi 
uzyskiwanie submetrowej dokładności absolutnej pomiarów kodowych bez użycia systemów augmentacyjnych. Podsumowując powyższe rozwarzania, wzrost dokładności pozycjonowania 
przyczyni się do lepszej jakości automatycznego prowadzenia maszyn rolniczych.\newline
\indent Systemy GNSS wyznaczją pozycję w sensie abolutnym z określoną dokładnością, która jest w przybliżeniu stała w czasie. Dlatego znakomitym 
uzupełnieniem absolutnego pozycjonowania GNSS jest korzystanie z sensorów elektronicznych takich jak żyroskopy i akcelerometry pozwalające na 
pomiar przemieszczenia obiektów z bardzo wysoką dokładnością w krótkich odstępach czasowych\footnote{np. w czsie równym interwałowi otrzymywania rozwiązania z GNSS}
względem znanej początkowej pozycji. Ponadto powyższe urządzenia pozwalają na wyznaczanie orientacji przestrzennej oraz jej zmian, tak bardzo potrzebne w celu 
ustalania optymalnego kursu oraz kompensacji pozycji GNSS ze względu na nierówności terenu. Wykorzystanie narzędzi oraz technik nawigacji inercjalnej, które w połączeniu 
z pozycjonowaniem absolutnym GNSS - tzw. fuzja danych pozwoli na znaczne ulepszenie dokładności prowadzenia maszyn rolniczych. Wobec powyższych słuszne zatem wydaje się 
stwierdzenie, że algorytmy fuzji danych łączące dane z kilku źródeł pozycjonowania oparte na logice rozmytej bądź filtracji kalmana będą w przyszłości podstawą
pozycjonowania w rolnictwie precyzyjnym. Podsumowując ten akapit warto dodać, że algorytmy fuzji danych wykorzystujące pomiary przemieszczenia obiektów na podstawie 
sensorów inercjalnych, algorytmów przetwarzających obraz z kamer wideo umieszczonych na pojazdach, czy czujników zliczających obroty poszczególnych kół\footnote{
Nawigacja zliczeniowa}, są w stanie zredukować błąd pozycji GNSS o co najmniej $\frac{2}{3}$. W takim wypadku jeżeli spojrzymy na przewidywane polepszenie dokładności 
rozwiązania GNSS bazującego na obsewacjach kodowych otrzymamy dokładność absolutną na poziomie około 30cm i względną na poziomie kilku centymetrów. 
Taka dokładność pozwalająca na wykonywanie wszystkich zabiegów agrotechnicznych z wyjątkiem tych wymagających centymetrowej dokładności jak np. sadzenie, osiągnięta 
stosunkowo niewielkim kosztem\footnote{odbiorniki GNSS nie analizujące obserwacji fazowych są zregóły dużo tańsze} będzie zachęcała rolników do inwestowania 
w narzędzia rolnictwa precyzyjnego a tym samym pośrednio wspierała jego rozwój.\newline
\indent	Rozwój rolnictwa precyzyjnego nastąpi również pośrednio na podstawie postępu w naukach informatycznych oraz w elektronice. Możliwości 
obliczeniowe komputerów oraz jakość i dokładność czujników elektronicznych, na których bazuje nawigacja inercjalna czy zliczeniowa podwaja się średnio co dwa lata\footnote{
na podstawie empirycznego prawa Moore'a}. Ponieważ systemy prowadzenia maszyn relnictwa precyzyjnego są w znacznej mierze oparte na szeroko pojętej elektronice, zatem 
rolnictwo precyzyjne również pośrednio na podstawie powyższego faktu wykona krok do przodu. Z powodu oczywistej inercji nie będzie to może ktok aż tak wielki ale jednak
dostrzegalny.\newline
\indent Przytoczone powyżej perspektywy rozwoju, zwłaszcza modernizacja oraz budowa systemów satelitarnych GNSS, z racji ich kosztowności będą wymagały 
czasu. Zatem obecnie oraz w najbliższej przyszłości rozwój rolnictwa precyzyjnego wspierany będzie przez lokalne systemy augmentacyjne takie jak polska 
aktywna sieć geodezyjna ASG-EUPOS, transmitujące dokładne poprawki sygnału GNSS: kodowe w technlogii DGPS oraz fazowe w technologii RTK.
Na chwilę obecną w nawigacji satelitarnej a co za tym idzie również w rolnictwie precyzyjnym lokalne systemy augmentacyjne są niezbędne w celu osiągnięcia 
centymetrowych dokładności międzyprzejazdowych. Maksymalne dokładności pozycjonowania absolutnego w czasie rzeczywistym za pomocą technologii PPP jakie możemy
uzyskać dzisiaj są na poziomie decymetrowym. Dzieje się tak dlatego, że opuźnienia w dostępie do precyzyjnych produktów IGS, takich jak orbity i parametry 
zegarów satelitów są niestety zbyt duże \cite[][strona 215]{ggos}.

\section{Wnioski}
O ogromnej istotności rolnictwa precyzyjnego świadczy choćby fakt wspomniany w rozdziale czwartym niniejszej pracy. Mianowicie nawożenie gleby ma czterdziesto 
procentowy wpływ na wielkość plonowania roślin, a zatem jego zmienny rozkład przestrzenny wydaje się kluczowy do osiągnięcia wymiernych korzyści ekonomicznych,
nie zatruwając przy tym wód gruntowych. Systemy nawigacji satelitarnej pozwalają na szybką akwizycję oraz zwiększenie rozdzielczości danych terenowych
opisujących właściwości glebowo rolnicze pól uprawnych. Podczas prac plowych aktualna pozycja GNSS jest wykorzystywana w celu obliczenia punktowej dawki nawozu.\newline
\indent Za pomocą nawigacji satelitarnej sporządza się mapy zapotrzebowania gleby w wodę\footnote{przy użyciu czujników mierzących wilgotnośc gleby wyposarzonych w 
odbiorniki GNSS} a następnie wykorzystuje się pozycjonowanie GNSS w precyzyjnym nawadnianiu gleby zgodnie z docelowym rozkładem przestrzennym dawek $H_2O$.
\indent Rozwój rolnictwa precyzyjnego na świecie będzie wynikał pośrednio z modernizacji istniejących  oraz powstawania nowych systemów satelitarnego pozycjonowania GNSS,
rozwoju elektronicznych czujników pomiarowych nawigacji inercjalnej MEMS, a zwłaszcza z opracowania zaawansowanych algorytmów fuzji danych opartych na filtrach Kalmana.
Rozwój zaawansowanych technologii przyczyni się do zmniejszenia kosztów technik już istniejących, a dzięki temu rolnictwo precyzyjne stanie się jeszcze bardziej
popularne a jego narzędzia tańsze i łatwiej dostępne.\newline
\indent Zintegrowane systemy nawigacyjne integrujące obserwacje nawigacji satelitarnej GNSS, nawigacji intercjalnej INS, czujników rejestrujących liczbę obrotów kół pojazdu
oraz sensorów wideo, oparte na filtracji kalmana bądź logice rozmytej znacznie podnoszą dokładność pozycjonowania. Niestety firmy komercyjne
nie wykorzystują jednak tej techniki, a przynajmniej nie informują o tym w broszurach reklamowych, co przy niedostępności dokumentacji technicznej nie pozwala na 
stwierdzenie wykorzystania fuzji danych komercyjnie w czasie obecnym. Pomimo dużego zainteresowania społeczności akademickiej oraz bogatą literaturą dotyczącą opisanej powyżej 
fuzji danych, brak implementacji ze strony rynku komercyjnego jest co najmniej zastanawiający.\newline
\indent Na podstawie analizy wykorzystywanych algorytmów GNSS przeprowadzonej w rozdziale piątym, można wywnioskować, że systemy nawigacji satelitarnej 
są wykorzystywane w pełni. Firmy komercyjne w swoich ofertach precyzyjnego prowadzenia pojazdów oferują rozwiązania o bardzo szerokiej gamie dokładnościowej,
począwszy od rozwiązań kodowych wspomaganych korekcjami EGNOS poprzez precyzyjne pozycjonowanie w czasie rzeczywistym - kinematyczne PPP z poprawkami OmniSTAR XP, G2, HP, na 
bardzo dokładnym pozycjonowaniu opartym na techologii różnicowego pozycjonowania RTK kończąc. Można zatem powiedzieć, że w przypadku zastosowania systemów satelitarnych 
wszelkie nowinki techniczne implementowane na uniwersytetach są natychmiast wdrażane przez firmy na ryunek komercyjny, niestety z jednym wyjątkiem - fuzja danych.\newline
\indent Jeżeli rozpatrujemy rynek polski, to przy średniej wielkości gospodarstw rolnych na poziomie 10.48 ha \cite[]{ARIMR} zakup
precyzyjnego dwuczęstotliwościowego odbiornika GNSS jest ekonomicznie niemożliwy. Za powyższy stan rzeczy, mianowicie zniszczenie Polskiego Ziemiaństwa 
poprzez usankcjonowaną tzw. reformą rolną kolektywizację gruntów, odpowiedzialność powinna ponieść okupująca Polskę w latach 1944 - 1989 władza warszawska.
Pod znakiem zapytania należy postawić uczciwość rządów III RP, które nie zrekompensowały bezprawnie zagarniętego majątku spadkobiercom polskich ziemian, 
a przecierz do dzisiaj w posiadaniu Agrncji Nieruchomości Rolnych Skarbu Państwa jest znaczna ilość ziemi pozwalająca na udzielenie choćby częsciowej rekompensaty. 
Obecnie niemożliwe jest zatem na szeroką skalę wdrożenie precyzyjnych technik prowadzenia pojazdów w Polsce.
Nadzieję stwarzają algorytmy fuzji danych z kilku niezależnych systemów nawigacyjnych. Dokładnść każdego z nich osobno nie musi być bardzo wysoka,
ale gdy obserwacje opracujemy łącznie za pomocą filtra kalmana wtedy nawet odbiorniki GNSS gorszej klasy mogą okazać się wystarczające.
Dla nielicznych korzystających z zaawansowanych dokładnościowo rozwiązań rolnictwa precyzyjnego nieocenionym wsparciem zapewne jest polska aktywna sieć
geodezyjna ASG-EUPOS.




