%uzasadnienie wyboru tematu 
%
% uzasadnienie wyboru tematu
% cel pracy ( przybliżenie, zapoznanie czytelnika, na czym się skoncentrowano a co pominięto i dlaczego)
% jakiego tematu dotycvzy praca czym się w niej zajęto
% ograniczniki ( które fragmanty świadomie zawężono)
% tutaj chyba coś o rolnictwie precyzyjnym. Może kilka roboczych definicji :)
W obecnym czasie technologia produkcji maszyn bardzo dynamicznie się rozwija. 
Maszyny stają się coraz większe oraz bardziej wydajne. 
Ponadto bardzo często wyposażone są w nowoczesne urządzenia elektroniczne. 
Niestety rolnicy nie mogą w pełni korzystać z rozwoju techniki. 
Natura ludzka wymusza na farmerach aby odpoczywali w nocy. Maszyny takim ograniczeniom nie podlegają.
Wobec powyższych można zaryzykować stwierdzenie, że to operator maszyn staje się granicą która limituje wzrost wydajności wykonywanej pracy.
Wynika z tego ogromna potrzeba posiadania zaawansowanych systemów automatycznego sterowania maszynami rolniczymi,
w celu zwiększenia wydajności oraz precyzji wykonywanej pracy \cite{CCTA_769_775}. 
Ponadto wykorzystanie wyspecjalizowanych narzędzi rolnictwa precyzyjnego zmniejsza koszty produkcji,
zmniejsza negatywny wpływ środków ochrony roślin i nawozów na środowisko naturalne, 
oraz przyczynia się do wzrostu jakości oraz plonowania płodów rolnych \cite{CCTA_943_950}.

TODO trzeba dokończyć wstęp.

