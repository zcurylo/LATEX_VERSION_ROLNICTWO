%uzasadnienie wyboru tematu 
%
% uzasadnienie wyboru tematu
% cel pracy ( przybliżenie, zapoznanie czytelnika, na czym się skoncentrowano a co pominięto i dlaczego)
% jakiego tematu dotycvzy praca czym się w niej zajęto
% ograniczniki ( które fragmanty świadomie zawężono)
% tutaj chyba coś o rolnictwie precyzyjnym. Może kilka roboczych definicji :)
W obecnym czasie technologia produkcji maszyn bardzo dynamicznie się rozwija. 
Maszyny stają się coraz większe oraz bardziej wydajne. 
Ponadto bardzo często wyposażone są w nowoczesne urządzenia elektroniczne. 
Niestety rolnicy nie mogą w pełni korzystać z rozwoju techniki. 
Natura ludzka wymusza na farmerach aby odpoczywali w nocy. Maszyny takim ograniczeniom nie podlegają.
Wobec powyższych można zaryzykować stwierdzenie, że to operator maszyn staje się granicą która limituje wzrost wydajności wykonywanej pracy.
Wynika z tego ogromna potrzeba posiadania zaawansowanych systemów automatycznego sterowania maszynami rolniczymi,
w celu zwiększenia wydajności oraz precyzji wykonywanej pracy \cite{CCTA_769_775}. 
Ponadto wykorzystanie wyspecjalizowanych narzędzi rolnictwa precyzyjnego zmniejsza koszty produkcji,
zmniejsza negatywny wpływ środków ochrony roślin i nawozów na środowisko naturalne, 
oraz przyczynia się do wzrostu jakości oraz plonowania płodów rolnych \cite{CCTA_943_950}.\\
\indent W pracy szczególny nacisk połorzono na przegląd zastosowania nowoczesnych systemów globalnego pozycjonowania GNSS w rolnictwie precyzyjnym 
zarówno na szczeblu naukowym jak i komercyjnym. W rozdziale drugim omówiono istotność globalnego układu odniesienia ITRF dla geodezji kosmicznej a tym samym pośrednio 
dla rolnictwa. Opisano także najważniejsze cechy systemów satelitarnego pozycjonowania GNSS, techniki oraz algorytmy pozycjonowania. Opisana została także 
integracja nawigacji satelitarnej z nawigacją inercjalną (GNSS + INS). W rozdziale trzecim opisano pokrótce algorytmy prowadzenia pojazdów rolniczych na podstawie 
danych nawigacyjnych. Rozdział czwarty koncentruje się na przeglądzie aktualnych badań naukowych, natomiast w rozdziale piątym opisano aktualny stopień wdrożenia 
rozwiązań wypracowanych przez świat uniwersytecki na rynku komercyjnym. Rozdział szósty zawiera perspektywy rozwoju oraz wnioski.\\
\indent Rozwój rolnictwa precyzyjnego na świecie będzie wynikał pośrednio z modernizacji istniejących  oraz powstawania nowych systemów satelitarnego pozycjonowania GNSS,
rozwoju elektronicznych czujników pomiarowych nawigacji inercjalnej MEMS, a zwłaszcza z opracowania zaawansowanych algorytmów fuzji danych opartych na filtrach Kalmana. 
Rozwój zaawansowanych technologii przyczyni się do zmniejszenia kosztów technik już istniejących, a dzięki temu rolnictwo precyzyjne stanie się jeszcze bardziej 
popularne a jego narzędzia tańsze i łatwiej dostępne.\\
\indent Jeżeli rozpatrujemy rynek polski, to przy średniej wielkości gospodarstw rolnych na poziomie 10.48 ha \cite[]{ARIMR} zakup 
precyzyjnego dwuczęstotliwościowego odbiornika GNSS jest ekonomicznie niemożliwy. Obecnie niemożliwe jest zatem na szeroką skalę wdrożenie precyzyjnych technik 
prowadzenia pojazdów. Nadzieję stwarzają algorytmy fuzji danych z kilku niezależnych systemów nawigacyjnych. Dokładnść każdego z nich osobno nie musi być bardzo wysoka,
ale gdy obserwacje opracujemy łącznie za pomocą filtra kalmana wtedy nawet odbiorniki GNSS gorszej klasy mogą okazać się wystarczające.
Dla nielicznych korzystających z zaawansowanych dokładnościowo rozwiązań rolnictwa precyzyjnego nieocenionym wsparciem zapewne jest polska aktywna sieć 
geodezyjna ASG-EUPOS.

